\documentclass[a4paper]{article}

\usepackage[T1]{fontenc}
\usepackage[ngerman]{babel}
\usepackage{lmodern}
\usepackage{fullpage}
\usepackage{parskip}
\usepackage{amsmath}
\usepackage{amsfonts}
\usepackage{tabularx}
\usepackage{graphicx}
\usepackage{enumitem}
%\usepackage{minted}
\usepackage{fontspec}
\usepackage{unicode-math}
\usepackage{luacode}
\usepackage{hyperref}
\usepackage{glossaries}

\usepackage{xcolor}
\usepackage{tikz}
\usetikzlibrary{arrows}

\title{Spezielle Themen der Informatik - Lösungen}

\author{Philip Bötel, Lorenz Tonner}


\tikzset{
  treenode/.style = {align=center, inner sep=0pt, text centered,
    font=\sffamily},
  box/.style = {treenode, rectangle, black, font=\sffamily\bfseries, draw=black, fill=white, thick,minimum height = 8mm},
}
\newcommand{\block}[2]{\hspace{2mm}\color{red}(\color{black}#1\color{red})\color{black} #2\hspace{2mm}}

\makeglossaries

\begin{document}
\maketitle
    
\begin{luacode}
    --Aufgabe 2
    function generateRandomAddition()
        x = math.random(1,100)
        y = math.random(1,100)
        return x .. " + " .. y .. " = " .. (x+y)
    end
    
    --Aufgabe 3
    function generateAdditionTabular()
        tex.print("\\begin{tabular} {c|c}")
        for line = 1, 2, 1 do
            for column = 1, 2, 1 do
                printCell()
                if column < 2 then
                    tex.print('&')
                end
            end
            tex.print('\\\\')
            if line < 2 then
                tex.print('\\hline')
            end
        end
        tex.print("\\end{tabular}")
    end

    function printCell()
        tex.print(generateRandomAddition())
    end
    
    --Aufgabe 4
    function generateDynamicAdditionTabular(size)
        tex.print("\\begin{tabular} {c|c} ")
        for line = 1, size, 1 do
            for column = 1, 2, 1 do
                printCell()
                if column < 2 then
                    tex.print('&')
                end
            end
            tex.print('\\\\')
            if line < size then
                tex.print('\\hline')
            end
        end
        tex.print("\\end{tabular}")
    end

    function printCell()
        tex.print(generateRandomAddition())
    end
\end{luacode}


\section{Hello World Macro}
    Erstellen sie ein Neues Tex-Makro mit dem Namen \textbackslash helloworld, welches \textit{Hello World!} ausgibt.
    
    \newcommand{\helloworld}[0]{\luadirect{tex.print("Hello World!")}}
    \helloworld
    
\section{Zufällige Addition}
    Erstellen Sie eine Lua-Funktion, welche eine Zufällige Additionsaufgabe zurück gibt, die Summanden sollten im Bereich von 1 bis 100 liegen. Der Rückgabewert sollte folgendermaßen ausschauen: \\ \textit{6 + 7 = 15} \\
    
    \luadirect{tex.print(generateRandomAddition())}
    
    
    
\section {Additionsaufgaben als Tabelle}
    Erstellen Sie eine Lua-Funktion, welche eine Tabelle mit Additionsaufgaben erstellt und ausgibt, die Summanden sollten im Bereich von 1 bis 100 liegen. Die Tabelle sollte Folgendermaßen ausschauen: \\\\
    \begin{tabular} {c|c}
        2 + 4 = 6 & 1 + 3 = 4\\
        \hline
        1 + 2 = 3 & 23 + 10 = 33\\
    \end{tabular}
    
    \luadirect{generateAdditionTabular()}
    
    
\section{Bonus: Additionsaufgaben als dynamische Tabelle}
    Erweitern Sie die Funktion aus der vorigen Aufgabe, sodass die Anzahl der Zeilen dynamisch generiert werden können. Erstellen Sie dazu ein TeX-Makro welches die Anzahl der Zeilen übergeben wird und die Lua-Funktion ausführt.
   
    \newcommand{\generateDynamicAdditionTabular}[1]{\luadirect{generateDynamicAdditionTabular(#1)}}
    
    \generateDynamicAdditionTabular{5}
\end{document}
